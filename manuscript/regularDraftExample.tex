%Title  :   Example using LaTeX template for creating a manuscript
%Author :   Vikesh Siddhu
%Date   :   25 March 2020
%email  :   vsiddhu@protonmail.com
%
\input regularDraftTemplate.tex

\begin{document}

\begin{center}
{\bf Title} \bsk\\
    Name~\footnote{email1@website1}~\textsuperscript{1} 
    and Possibly others~\footnote{email2@website2}~\textsuperscript{2}\\
{\em \textsuperscript{1} Institution 1.}\\
{\em \textsuperscript{2} Institution(s)}\\
Date: March 25, 2021
\end{center}

\begin{abstract}
Words
\end{abstract}

\section{Section}
\label{Label1}
\xb
\outl{Outline for this section will not appear in pdf unless changes are made
in the OUTLINE IN TEXT part of the template}
\xa

Some more words.

\subsection{Sub-Section}
\label{Label2}
\xb
\outl{Outline for this sub-sections}
\xa

Words again. We cite a paper~\cite{AuthorYear} in {\tt aBibFile.bib} using
BibTeX. To compile this file run {\tt pdflatex} then {\tt biblatex} and
then {\tt pdflatex} two more times.

\begin{equation}
    a = b + c
    \label{anEqn}
\end{equation}
%
Cite this equation later using~\eqref{anEqn} and refer to sections
using~\ref{AppASub}

%%%%%%%%%%%%%%%%%%%%%%%%%%%%%%%%%%%%%%%%%%%%%%%%%%%%%%%%%%%%%%%%%%%%%%%%%%%%%%%%
%%%%%%%%%%%%%%%%%%%%%%%%%%%% REFERENCES %%%%%%%%%%%%%%%%%%%%%%%%%%%%%%%%%%%%%%%%
%%%%%%%%%%%%%%%%%%%%%%%%%%%%%%%%%%%%%%%%%%%%%%%%%%%%%%%%%%%%%%%%%%%%%%%%%%%%%%%%

\bibliography{aBibFile}
\bibliographystyle{unsrt}

%%%%%%%%%%%%%%%%%%%%%%%%%%%%%%%%%%%%%%%%%%%%%%%%%%%%%%%%%%%%%%%%%%%%%%%%%%%%%%%%
%%%%%%%%%%%%%%%%%%%%%%%%%%%%%% APPENDIX %%%%%%%%%%%%%%%%%%%%%%%%%%%%%%%%%%%%%%%%
%%%%%%%%%%%%%%%%%%%%%%%%%%%%%%%%%%%%%%%%%%%%%%%%%%%%%%%%%%%%%%%%%%%%%%%%%%%%%%%%

%Renew Equation Counter 
\setcounter{equation}{0}
\renewcommand{\theequation}{\Alph{section}\arabic{equation}}
\counterwithin*{equation}{section}

%
%%Prefix sections numbers with Appendix-
\renewcommand{\thesection}{Appendix~\Alph{section}}
\renewcommand{\thefigure}{\Alph{figure}}
%

%Renew Section Counter 
\setcounter{section}{0}

\section{Section}

\begin{equation}
    c = d + e
    \label{Eqn2}
\end{equation}
%
\subsection{Sub-section}
\label{AppASub}

\section{New Section}

\begin{equation}
    k := k_1 k_2
    \label{EqnD}
\end{equation}

\end{document}
