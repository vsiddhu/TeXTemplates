%Title  :   LaTeX template for creating a manuscript
%Author :   Vikesh Siddhu
%Date   :   25 March 2020
%email  :   vsiddhu@protonmail.com
%
\documentclass[10pt]{article}
\usepackage{caption}

\usepackage{subcaption}
\usepackage[utf8]{inputenc}

%To adjust how the equation counters change
\usepackage{chngcntr}

%Math related packages
\usepackage{amssymb}
\usepackage{amsfonts}
\usepackage{amsmath}
\usepackage{amsthm}
\usepackage{mathtools}

\usepackage{graphicx}
\usepackage{indentfirst}
\usepackage{cite}
\usepackage{enumerate}
\usepackage{url}

%Font settings
\usepackage{lmodern}
\usepackage{newpxtext}

%Fancy footnote
\usepackage[symbol]{footmisc}
\renewcommand{\thefootnote}{\fnsymbol{footnote}}

%Tiks for figures and colors for Citations
\usepackage{tikz}
\definecolor{darkblue}{rgb}{0.15,0.35,0.55}
\definecolor{reddish}{rgb}{.8, 0.2, 0.2}

%Hyperreff is best loaded last, it can conflict with other packages
%Setting hypertexnames to false avoids come issues with \label command
\usepackage[pdfpagelabels, linktocpage=true, hypertexnames=false]{hyperref}
\hypersetup{
colorlinks=true,
citecolor=darkblue,
linkcolor=reddish,
urlcolor=darkblue,
pdfauthor={},
pdftitle={},
pdfsubject={}
}
%%%%%%%%%%%%%%%%%%%%%%%%%%%%%%END OF PACAKGE LIST%%%%%%%%%%%%%%%%%%%%%%%%%%%%%%

%Set Margins
\oddsidemargin 0 cm
\evensidemargin 0 cm
\topmargin -1.5 cm \textheight 23 cm \textwidth 16.5 cm
\raggedbottom

%Load commands defined using amsmath
\input vsCom.sty

%Commands defined using amsthrm
\theoremstyle{remark}
\newtheorem*{theorem}{Theorem}
\newtheorem{corollary}{Corollary}
\renewcommand\qedsymbol{$\blacksquare$}

%%%%%%%%%%%%%%%%%%%%% Commands for OUTLINE IN TEXT %%%%%%%%%%%%%%%%%%%%%%%%%%%%

%INSTRUCTIONS. 
%- Default is text with outlines, and \ca, \cb (defined with amsmath commands) 
%  are used to comment out revisions in definitions for other cases. 
%
%- TEXT, NO OUTLINE:  \ca -> %\ca, \cb -> %\cb below 'TEXT ALONE'. 
%- OUTLINE, NO TEXT:  \ca -> %\ca, \cb -> %\cb below 'OUTLINE ALONE'
%- Newpage. \np for text alone, \nq for text + outline, \nr for outline alone
%- Material to be EXCLUDED for OUTLINE ALONE is between \xa and \xb. 
%- Presence of \xb elsewhere does not matter, as it is ignored.
% 
% DEFINITIONS
\def\outl#1{\par{\medskip\noindent\hspace*{0.1cm}\bf
      \mathversion{bold}#1\mathversion{normal}\smallskip} }
\def\np{} \def\nq{\newpage } \def\nr{} \def\xa{} \def\xb{} \def\xn{} \def\xp{}

% TEXT ALONE USE %\ca %\cb. INCLUDE OUTLINE: \ca + \cb.  
%\ca
 \def\outl#1{}\def\np{\newpage }\def\nq{}\def\xa{}\def\xb{}\def\xn{}\def\xp{}
%\cb

% OUTLINE ALONE USE %\ca %\cb, OTHERWISE \ca \cb
\ca
 \def\outl#1{\par{\medskip\noindent\hspace*{.5cm}\bf
      \mathversion{bold}#1\mathversion{normal}\smallskip} }
 \long\def\xa#1\xb{} %This comments out segments from \xa to the next \xb
 \def\nq{} \def\nr{\newpage }\def\xn{\nopagebreak }\def\xp{\pagebreak }
\cb

%%%%%%%%%%%%%%%%%%%% END OUTLINE IN TEXT %%%%%%%%%%%%%%%%%%%%%%%%%%%%%%%%%%%%%%%

